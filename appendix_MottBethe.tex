\chapter{Derivation of the Mott-Bethe Formula}
\label{ap: MottBethe}

The Mott-Bethe formula is often used to convert in between
the X-ray scattering factor
%
\begin{equation}
  f_\text{X}(\boldsymbol{q}) = \int n_\text{e}(\boldsymbol{r}) \mathrm{e}^{ - \mathrm{i} \boldsymbol{q} \cdot \boldsymbol{r}} \mathrm{d}^3 r
\end{equation}
%
and the electron scattering factor
%
\begin{equation}
  f_\text{e}(\boldsymbol{q}) = - \frac{m_\text{e}}{2 \unslant[-.2]\pi \hbar^2} \int U(\boldsymbol{r}) \mathrm{e}^{ - \mathrm{i} \boldsymbol{q} \cdot \boldsymbol{r}} \mathrm{d}^3 r
\end{equation}
%
for the purpose of calculations.
However, the source of this handy relationship is rarely stated.
Thus, for the completeness of this thesis,
the derivation for this equation is described below,
as laid out in Ref.~\cite{LandauLifshitzBook}.

During a scattering event with an atom, X-rays interact mainly with
the electron number density $n_\text{e}(\boldsymbol{r})$
while electrons are deflected by the electrostatic potential $\phi(\boldsymbol{r})$
caused by the bound electrons and the nucleus.
%
A starting point to relate these two quantities is the Poisson Equation
and the total atomic charge density~$\rho(\boldsymbol{r})$,
%
\begin{equation}
  \nabla^2 \phi(\boldsymbol{r}) = -\frac{\rho(\boldsymbol{r})}{\epsilon_0}
\end{equation}
%
in which their Fourier transforms are inserted,
%
\begin{equation}
  \begin{aligned}
    \nabla^2 \left( \int \phi_{\boldsymbol{q}} \mathrm{e}^{ \mathrm{i} \boldsymbol{q} \cdot \boldsymbol{r} } \mathrm{d}^3 q \right)
      & = -\frac{1}{\epsilon_0} \left( \int \rho_{\boldsymbol{q}} \mathrm{e}^{\mathrm{i} \boldsymbol{q} \cdot \boldsymbol{r}} \mathrm{d}^3 q \right) \\
    \int \phi_{\boldsymbol{q}} \left( -q^2 \mathrm{e}^{\mathrm{i} \boldsymbol{q} \cdot \boldsymbol{r}} \right) \mathrm{d}^3 q
      & = -\frac{1}{\epsilon_0} \int \rho_{\boldsymbol{q}} \mathrm{e}^{\mathrm{i} \boldsymbol{q} \cdot \boldsymbol{r}} \mathrm{d}^3 q \\
    \phi_{\boldsymbol{q}} & = \frac{1}{\epsilon_0 q^2} \rho_{\boldsymbol{q}}
  \end{aligned}
\end{equation}
%
Therefore,
%
\begin{equation}
  \begin{aligned}
    \int \phi(\boldsymbol{r}) \mathrm{e}^{ - \mathrm{i} \boldsymbol{q} \cdot \boldsymbol{r}} \mathrm{d}^3 r
      & = \frac{1}{\epsilon_0 q^2} \int \rho(\boldsymbol{r}) \mathrm{e}^{ - \mathrm{i} \boldsymbol{q} \cdot \boldsymbol{r}} \mathrm{d}^3 r
  \end{aligned}
\end{equation}
%
On the right side, $\rho(\boldsymbol{r})$ can be expanded into a point-like nuclear charge density
$\rho_\text{N}(\boldsymbol{r}) = Z e \delta^3(\boldsymbol{r})$ and an extended electronic charge density $\rho_\text{e}(\boldsymbol{r}) = e n_\text{e}(\boldsymbol{r})$,
%
\begin{equation}
  \begin{aligned}
    \frac{1}{\epsilon_0 q^2} \int \rho(\boldsymbol{r}) \mathrm{e}^{\mathrm{i} \boldsymbol{q} \cdot \boldsymbol{r}} \mathrm{d}^3 r
      & = \frac{1}{\epsilon_0 q^2} \int \left( Z e \delta^3(\boldsymbol{r}) - e n_\text{e}(\boldsymbol{r}) \right) \mathrm{e}^{\mathrm{i} \boldsymbol{q} \cdot \boldsymbol{r}} \mathrm{d}^3 r \\
      & = \frac{e}{\epsilon_0 q^2} \left( Z - \int n_\text{e}(\boldsymbol{r}) \mathrm{e}^{\mathrm{i} \boldsymbol{q} \cdot \boldsymbol{r}} \mathrm{d}^3 r \right) \\
      & = \frac{e}{\epsilon_0 q^2} \left( Z - f_\text{X}(\boldsymbol{q}) \right)
  \end{aligned}
\end{equation}
%
where $Z$ is the atomic number of the atom and $n_\text{e}(\boldsymbol{r})$ is the number density of the bound electrons.
%
On the left side, recall that $U(\boldsymbol{r}) = -e \phi(\boldsymbol{r})$,
%
\begin{equation}
  \begin{aligned}
    \int \phi(\boldsymbol{r}) \mathrm{e}^{ - \mathrm{i} \boldsymbol{q} \cdot \boldsymbol{r}} \mathrm{d}^3 r
      & = -\frac{1}{e} \int U(\boldsymbol{r}) \mathrm{e}^{ - \mathrm{i} \boldsymbol{q} \cdot \boldsymbol{r}} \mathrm{d}^3 r \\
      & = \frac{2 \unslant[-.2]\pi \hbar^2}{m_\text{e} e} f_\text{e}(\boldsymbol{q})
  \end{aligned}
\end{equation}
%
By recombining the result from the left and right sides, the Mott-Bethe formula is thus recovered:
%
\begin{equation}
  \begin{aligned}
    f_\text{e}(\boldsymbol{q}) = \frac{m_\text{e} e^2}{2 \unslant[-.2]\pi \hbar^2 \epsilon_0 q^2} \left( Z - f_\text{X}(\boldsymbol{q}) \right)
  \end{aligned}
\end{equation}
