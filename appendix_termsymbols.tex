
\chapter{Atomic and Molecular Terms Symbols}
\label{ap: term-symbols}

Due to Liveing, Dewar, and Hund~\cite{Liveing1915, Hund1927},
atomic orbitals are labelled conventionally
by $n$X, where $n$ is the principal quantum number and
X is lower-case letter denoting the value of the azimuthal quantum number~$\ell$;
for $\ell = 0, 1, 2, 3$, $\mathrm{X} = \mathrm{s}, \mathrm{p}, \mathrm{d}, \mathrm{f}$
where s = `sharp,' p = `principal,' d = `diffuse,' and f = `fundamental';
for higher values of $\ell$, X is the next unused letter in the latin alphabet.
%
A given electron configuration can then be specified by listing out
the orbitals in order of $n$ and $\ell$ with electron occupancy in superscript on the right,
e.g. (1s)$^2$(2s)$^2$(2p)$^6$(3s)$^2$(3p)$^6$(3d)$^6$(4s)$^2$ for iron.
%
In X-ray spectroscopy, the value of $n$ is denoted instead
by the upper-case letters K, L, M, N, and so on;
subshells are simply enumerated by a subscript~\cite{Jenkins1991}.

The electronic state of an atom is labelled by its term symbol,
usually using the Russell-Saunders scheme: $^{2S + 1}L_J$,
where $S, L, J$ are the total spin, orbital, and net angular momentum quantum numbers;
the value of $L$ is denoted by the upper-case letters S, P, D, F, and so on.
This nomenclature is named after American astronomer Henry Norris Russell (1877--1957)
and physicist Frederick Albert Saunders (1875--1963)~\cite{Russell1925}.

A different nomenclature, due to Robert S. Mulliken (1896--1986),
is used to label the electronic orbitals and states of molecules
since the operators associated the atomic quantum numbers no longer commute
with the total Hamiltonian.
Instead, they are labelled by the Mulliken symbol~$\Gamma$ of the irreducible representation
corresponding to their symmetry~\cite{Mulliken1955},
where $\Gamma = \mathrm{A}$ or B for those symmetric or anti-symmetric under rotation,
E or T if they are doubly or triply degenerate;
a `1' or `2' and `g' or `u' to the right in subscript denotes symmetry or anti-symmetry
with respect to a perpendicular mirror plane and an inversion centre.
%
As before, the lower case is used for orbitals and the upper case is for states.

% Character table
In reference to the works involving octahedral [ML$_6$]$^{+n}$ complexes
in Chapter~\ref{ch: SCO}, the character table for the point group O\textsubscript{h}
is included here in Table~\ref{tab: Oh}.
%
\begin{table}[ht!]
  \centering
  {\renewcommand*{\arraystretch}{1.5}
  \begin{tabular}{ C{0.7cm} C{0.7cm} C{0.7cm} C{0.7cm} C{0.7cm} C{0.7cm} C{0.7cm} C{0.7cm} C{0.7cm} C{0.7cm} C{0.7cm} }
    \toprule
    $\Gamma$ & E & 8C\textsubscript{3} & 6C\textsubscript{2} & 6C\textsubscript{4}
      & 3C\textsubscript{2} & i & 6S\textsubscript{4} & 8S\textsubscript{6} & 3$\sigma$\textsubscript{h}
      & 6$\sigma$\textsubscript{d} \\
    \midrule
    A\textsubscript{1g} & 1 & 1 & 1 & 1 & 1 & 1 & 1 & 1 & 1 & 1 \\
    A\textsubscript{2g} & 1 & 1 & -1 & -1 & 1 & 1 & -1 & 1 & 1 & -1 \\
    E\textsubscript{g} & 2 & -1 & 0 & 0 & 2 & 2 & 0 & -1 & 2 & 0 \\
    T\textsubscript{1g} & 3 & 0 & -1 & 1 & -1 & 3 & 1 & 0 & -1 & -1 \\
    T\textsubscript{2g} & 3 & 0 & 1 & -1 & -1 & 3 & -1 & 0 & -1 & 1 \\
    A\textsubscript{1u} & 1 & 1 & 1 & 1 & 1 & -1 & -1 & -1 & -1 & -1 \\
    A\textsubscript{2u} & 1 & 1 & -1 & -1 & 1 & -1 & 1 & -1 & -1 & 1 \\
    E\textsubscript{u} & 2 & -1 & 0 & 0 & 2 & -2 & 0 & 1 & -2 & 0 \\
    T\textsubscript{1u} & 3 & 0 & -1 & 1 & -1 & -3 & -1 & 0 & 1 & 1 \\
    T\textsubscript{2u} & 3 & 0 & 1 & -1 & -1 & -3 & 1 & 0 & 1 & -1 \\
    \bottomrule
  \end{tabular}
  }
  \caption{Character table for the O\textsubscript{h} point group~\cite{Harris1978}.}
  \label{tab: Oh}
\end{table}
