
\chapter{Crystal Field Splitting Energy}
\label{ap: cft}

Table~\ref{tab: cft-full} lists representative values of
the crystal field splitting energy~$\Delta_\text{oct}$
for octahedral complexes of the form $\mathrm{[ML_6]}^{n+}$
for monodentate ligands L and $\mathrm{[ML_3]}^{n+}$ for bidentate ones.
The ligands are sorted according to the spectrochemical series,
in increasing order of ligand field strength~\cite{Ryutaro1938a, Ryutaro1938b, FiggisBook}.
%
\begin{table}[ht!]
  \centering
  {\renewcommand*{\arraystretch}{1.5}
    \begin{tabular}{ c c c c c c c c c c }
      \toprule
      \multirow{2}{*}{
        \begin{minipage}[c]{1.75cm} \centering Electron System \end{minipage}
      } & \multirow{2}{*}{Ion}
      & \multicolumn{8}{c }{$\Delta_\text{oct}$ ($\times  10^3$~cm$^{-1}$)} \\
      \cline{3-10}
       &  & Br$^-$ & Cl$^-$ & F$^-$ & ox$^{2-}$ & OH$_2$ & NH$_3$ & en & CN$^-$ \\
      \midrule
      d$^1$ & Ti$^{3+}$ &  & 13 & 18.9 &  & 20.1 & 17 &  & 22 \\
       & V$^{4+}$ &  & 15 & 20.1 &  &  &  &  &  \\
      d$^2$ & V$^{3+}$ &  & 12 & 16.1 & 18 & 20 & 18 &  & 24 \\
      d$^3$ & V$^{2+}$ &  & 8 &  &  & 12.4 &  &  &  \\
       & Cr$^{3+}$ & 13 & 13 & 14.5 & 17.4 & 17 & 21.5 & 22 & 26 \\
       & Mo$^{3+}$ & 14.5 & 19.2 &  &  &  &  &  &  \\
      d$^4$ & Cr$^{2+}$ &  & 10 & 14 &  & 13 &  & 18 &  \\
       & Mn$^{3+}$ &  & 17.5 & 22 & 20 & 20 &  &  & 31 \\
      d$^5$ & Mn$^{2+}$ & 7 & 7.5 & 7.8 &  & 8.5 &  &  & 33 \\
       & Fe$^{3+}$ &  &  & 14 & 14 & 14 &  &  & 35 \\
      d$^6$ & Fe$^{2+}$ &  &  & 10 &  & 10 &  &  & 32 \\
       & Co$^{3+}$ &  &  & 13 & 18 & 20.8 & 22.9 & 23.2 &  \\
       & Ru$^{2+}$ &  &  &  &  & 19.8 & 28.1 &  &  \\
       & Rh$^{2+}$ & 19 & 20.4 &  & 26 & 27 & 34 & 35 &  \\
       & Ir$^{3+}$ & 23 & 25 &  &  &  & 41 & 41 &  \\
       & Pt$^{4+}$ & 25 & 29 & 33 &  &  &  &  &  \\
      d$^7$ & Co$^{2+}$ & 6.5 & 7.65 & 8.3 & 11 & 9.3 & 10.2 & 11 &  \\
      d$^8$ & Ni$^{2+}$ & 6.8 & 7.2 & 7.25 &  & 8.5 & 11.5 & 11.5 &  \\
      d$^9$ & Cu$^{2+}$ &  &  &  &  & 12 & 16 & 16 & \\
      \bottomrule
    \end{tabular}
  }
  \caption{Octahedral CFT splitting energy~$\Delta_\text{oct}$
    of some transition-metal complexes of the form [ML\textsubscript{6}]\textsuperscript{n+}
    for all monodentate ligands L. Ligands $\mathrm{L = ox^{2-}}$ and en are bidentate
    and form complexes of the form $\mathrm{[ML_3]}^{n+}$.
    Note that $\mathrm{ox^{2-} = (COO)_2^{2-}}$ is the oxalate dianion
    and $\mathrm{en = C_2 H_4 (NH_2)_2}$ is ethylenediamine.
    Data herein is taken from Ref.~\cite{FiggisBook}.
  }
  \label{tab: cft-full}
\end{table}

Note that values of $\Delta_\text{oct}$ also follow an increasing trend as a function
of the oxidation state and period number of the metal cation,
e.g. $\Delta_\text{oct}(\mathrm{Fe^{3+}}) < \Delta_\mathrm{o}(\mathrm{Fe^{2+}})$
and $\Delta_\mathrm{o}(\mathrm{Fe^{2+}}) < \Delta_\mathrm{o}(\mathrm{Ru^{2+}})$.
Both effects are consequences of $\Delta_\text{oct} \propto \frac{1}{r^5}$,
where $r$ is the metal--ligand interatomic distance;
a more highly charged metal cation pulls its ligands closer while
a larger one allows them to approach by lessening steric hindrance~\cite{FiggisBook}.
