
\chapter{Theory of Spin Crossover}
\label{ap: sco}

In Section~\ref{sec: SCO-theory}, the theory of thermal and photoinduced
spin crossover is discussed.
%
In particular, the non-adiabatic multiphonon relaxation model
by Buhks et al~\cite{Buhks1980} is mentioned in the context of
the LS$\leftarrow$HS relaxation following LIESST.
%
Here, this model is described in more detail for reference.

Following the lead of Hauser in Ref.~\cite{SCO-II},
consider Fermi's Golden Rule~\cite{Dirac1927, Fermi1950},
%
\begin{equation}
  \begin{aligned}
    w_{i \rightarrow f} & = \frac{2 \unslant[-.2]\pi}{\hbar} |V_{i f}|^2 \rho_f
  \end{aligned}
  \label{eq: fermi-golden}
\end{equation}
%
where $w_{i \rightarrow f}$ is the probability of transition from an initial state~$i$
(in the $m$-th vibrational level of the HS state)
to a final state~$f$ (in the $m^\prime$-th vibrational level of the LS state),
$|V_{i f}|^2$ is the matrix element of the perturbing potential
that couples states $i$ and $f$, and $\rho_f$ is the density of states at $f$.

Given that the perturbing potential is
the spin--orbit interaction~$\hat{H}_\text{SO} = \zeta \hat{\boldsymbol{L}} \cdot \hat{\boldsymbol{S}}$,
the coupling matrix element can be expanded as
%
\begin{equation}
  \begin{aligned}
    |V_{i f}|^2 & = |\langle \Psi_f | \hat{H}_\text{SO} | \Psi_i \rangle|^2 \\
      & = |\langle \psi_\text{HS} | \hat{H}_\text{SO} | \psi_\text{LS} \rangle|^2
      | \langle \chi_{m^\prime} | \chi_m \rangle |^2
  \end{aligned}
\end{equation}
%
where $| \psi \rangle, | \chi \rangle $ are the electronic and vibrational parts
of the total wavefunction~$| \Psi \rangle$ and,
according to the Condon~approximation, $| \Psi_i \rangle = | \psi_\text{HS} \rangle | \chi_m \rangle$
and $| \Psi_f \rangle = | \psi_\text{LS} \rangle | \chi_{m^\prime} \rangle$.
%
To evaluate this quantity, consider that
the electronic states are mixed to some degree
due to the presence of the spin--orbit interaction~$\hat{H}_\text{SO}$
and thus need to be expressed using first-order perturbation theory,
%
\begin{equation}
  \begin{aligned}
    | \psi_\text{LS} \rangle
      & \approx | \psi_\text{LS}^{(0)} \rangle
        + \sum_{j}
          \frac{\langle \psi_j^{(0)} | \hat{H}_\text{SO} | \psi_\text{LS}^{(0)} \rangle}{E_\text{LS}^{(0)} - E_j^{(0)}} | \psi_j^{(0)} \rangle \\
    \langle \psi_\text{HS} |
      & \approx \langle \psi_\text{HS}^{(0)} |
        + \sum_{j}
          \frac{\langle \psi_\text{HS}^{(0)} | \hat{H}_\text{SO} | \psi_j^{(0)} \rangle}{E_\text{HS}^{(0)} - E_j^{(0)}} \langle \psi_j^{(0)} |
  \end{aligned}
\end{equation}
%
where the energy denominators are evaluated in
the equilibrium nuclear configuration of the respective states.
Then,
%
\begin{equation}
  \begin{aligned}
    \langle \psi_\text{HS} | \hat{H}_\text{SO} | \psi_\text{LS} \rangle
      & = \langle \psi_\text{HS}^{(0)} | \hat{H}_\text{SO} | \psi_\text{LS}^{(0)} \rangle
        + \sum_{j} \langle \psi_\text{HS}^{(0)} | \hat{H}_\text{SO} | \psi_j^{(0)} \rangle
        \langle \psi_j^{(0)} | \hat{H}_\text{SO} | \psi_\text{LS}^{(0)} \rangle \\
      & \quad \left( \frac{1}{E_\text{LS}^{(0)} - E_j^{(0)}} + \frac{1}{E_\text{HS}^{(0)} - E_j^{(0)}} \right) \\
      & = 0 + \langle \mathrm{^5T_{2g}} | \hat{H}_\text{SO} | \mathrm{^3T_{1g}} \rangle
        \langle \mathrm{^3T_{1g}} | \hat{H}_\text{SO} | \mathrm{^1A_{1g}} \rangle
        \left( \frac{1}{\Delta E_\text{LI}^{(0)}} + \frac{1}{\Delta E_\text{HI}^{(0)}} \right)
  \end{aligned}
\end{equation}
%
where $\Delta E_\text{LI}^{(0)}, \Delta E_\text{HI}^{(0)}$ are the energy difference
between the triplet intermediate state $\mathrm{^3T_{1g}}$ and the other states.
From Ref.~\cite{Griffith1964}, $\mathrm{^3T_{1g}}$,
with electronic configuration $\mathrm{(t_{2g})^5 (e_g^*)^1}$,
is the only term which has non-vanishing spin--orbit matrix elements
with both $\mathrm{^1A_{1g}}$ and $\mathrm{^5T_{2g}}$,
%
\begin{equation}
  \begin{aligned}
    \langle \mathrm{^5T_{2g}} | \hat{H}_\text{SO} | \mathrm{^1A_{1g}} \rangle
      & = 0 \\
    \langle \mathrm{^5T_{2g}} | \hat{H}_\text{SO} | \mathrm{^3T_{1g}} \rangle
      & = - \sqrt{6} \zeta \\
    \langle \mathrm{^3T_{1g}} | \hat{H}_\text{SO} | \mathrm{^1A_{1g}} \rangle
      & = \sqrt{3} \zeta
  \end{aligned}
\end{equation}

Assume that the LS and HS potentials are identical and harmonic of
the same frequency~$\omega$, with the latter displaced energetically by $\Delta E_\text{HL}^{(0)}$ and
configurationally by $\Delta Q_\text{HL} = \sqrt{6} \Delta r_\text{HL}$
along a single internal vibrational coordinate~$Q$, namely the totally symmetric metal--ligand stretch mode.
Then, energy conservation requires simply $m^\prime = m + n$,
where $n = \frac{\Delta E_\text{HL}^{(0)}}{\hbar \omega}$ is the reduced energy gap between the LS and HS states,
and the density of states~$\rho_f$ becomes $\frac{g_f}{\hbar \omega}$,
where $g_f = 1$ is the degeneracy of the final electronic state.

To obtain the LS$\leftarrow$HS relaxation rate constant~$k_\text{HL}(T)$,
Eq.~\eqref{eq: fermi-golden} is combined with those above and
ensemble-averaged over all $m$,
%
\begin{equation}
  \begin{aligned}
    k_\text{HL}(T)
      & = \frac{2 \unslant[-.2]\pi}{\hbar^2 \omega}
        |\langle \psi_\text{HS} | \hat{H}_\text{SO} | \psi_\text{LS} \rangle|^2 \bar{F}_n(T)
  \end{aligned}
\end{equation}
%
where $\bar{F}_n$ is the ensemble average of the Franck-Condon factor~$F_{m n}$,
%
\begin{equation}
  \begin{aligned}
    F_{m n}(T) & = | \langle \chi_{m + n} | \chi_m \rangle |^2 \\
    \bar{F}_n(T)
      & = \frac{\sum \limits_m F_{m n}(T) \mathrm{e}^{-m \hbar \omega / k_\text{B} T}}{\sum \limits_m \mathrm{e}^{-m \hbar \omega / k_\text{B} T}}
  \end{aligned}
\end{equation}

% Sousa, de Graaf calculations
In a similar procedure, Sousa et~al~\cite{Sousa2013} have evaluate
more spin--orbit coupling matrix elements using high-level quantum-chemical methods.
%
Table~\ref{tab: sco-so} shows some of these computational results.
%
\begin{table}[ht!]
  \centering
  {\renewcommand*{\arraystretch}{1.5}
    \begin{tabular}{| c | c | c c c c c c c c |}
      \cline{3-10}
      \multicolumn{2}{c|}{} & \multicolumn{8}{c |}{$\psi_f$} \\
      \cline{3-10}
      \multicolumn{2}{c|}{} & $\mathrm{^1 A_{1g}}$
        & $\mathrm{^1 T_{1g}}$ & $\mathrm{^1 MLCT}$ & $\mathrm{^3 T_{1g}}$
        & $\mathrm{^3 T_{2g}}$ & $\mathrm{^3 MLCT}$ & $\mathrm{^5 T_{2g}}$ & $\mathrm{^5 MLCT}$ \\
      \hline
      \multirow{8}{*}{$\psi_i$} & $\mathrm{^1 A_{1g}}$ &  &  &  & 527.7 & 83.7 & 81.6 & 0 & 0 \\
       & $\mathrm{^1 T_{1g}}$ &  &  &  & 75.5 & 131.4 & 164.7 & 0 & 0 \\
       & $\mathrm{^1 MLCT}$ &  &  &  & 96.0 & 214.3 & 199.9 & 0 & 0 \\
       & $\mathrm{^3 T_{1g}}$ & 527.7 & 75.5 & 96.0 &  &  &  & 417.7 &  \\
       & $\mathrm{^3 T_{2}}$ & 83.7 & 131.4 & 214.3 &  &  &  & 219.9 &  \\
       & $\mathrm{^3 MLCT}$ & 81.6 & 164.7 & 199.9 &  &  &  & 6.2 & 344.3 \\
       & $\mathrm{^5 T_{2g}}$ & 0 & 0 & 0 & 417.7 & 219.9 & 6.2 &  &  \\
       & $\mathrm{^5 MLCT}$ & 0 & 0 & 0 &  &  & 344.3 &  & \\
      \hline
    \end{tabular}
  }
  \caption{Select spin--orbit coupling matrix elements
    $\langle \psi_f | \hat{H}_\text{SO} | \psi_i \rangle$
    of $\mathrm{[Fe^{II}(bpy)_3]^{2+}}$ in the equilibrium nuclear configuration
    of the $\mathrm{^1 A_{1g}}$ LS state, calculated at the CASSCF/CASPT2 level~\cite{Sousa2013}.
  }
  \label{tab: sco-so}
\end{table}
