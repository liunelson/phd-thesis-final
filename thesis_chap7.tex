
\chapter{Conclusion}
\label{ch: conclusion}

% Scientific question
In Ch.~\ref{ch: intro}, I introduced the scientific question that
I had set out to answer in the course of my Ph.D. studies.
%
It pertained to the possibility of capturing the chemistry of \emph{large},
\emph{low-symmetry}, and \emph{weakly scattering} molecules in action
by making movies with femtosecond and sub-angstrom resolution
using ultrafast electron diffraction.
%
In the subsequent chapters, I described how this task was achieved experimentally
for five such systems: two Bechgaard-Fabre salts (Secs.~\ref{sec: UED-EDOPF6}
and~\ref{sec: UED-EDOSbF6}), a diarylethene derivative~(Ch.~\ref{ch: UED-DAE}),
and two spin-crossover complexes (Secs.~\ref{sec: UED-AZA} and~\ref{sec: UED-BPY}).
%
Indeed, the answer is \emph{yes}:
%
\vspace{0.5\baselineskip}
%
\begin{center}
  \begin{minipage}{0.8\textwidth}
    Molecular movies can be made of systems more complex than simple gases and metals.
  \end{minipage}
\end{center}

\vspace{0.5\baselineskip}

% Summary of conclusions
There is more to the works that are presented in this thesis than a mere experimental
demonstration of the capabilities of UED.
%
Much insight was gained into the structure--function correlation
that drives the material properties and photoresponse of the studied molecular systems.

% EDO
For the (EDO-TTF)$_2$X systems, the UED results provided support
for the {ex-PHH} description and uncovered the important role played by the counterions
(X = {PF$_6$}$^-$, {SbF$_6$}$^-$) --- heretofore thought to be mere spectators ---
in the charge fluctuations leading to the photoinduced transition
from the insulating ground state to a transient metallic state at low temperature.
%
Further work showed that singular value decomposition and global analysis
could be applied to project out the dominant atomic motions
from time-resolved diffraction data directly in reciprocal space.
Clearly, there are only a few key modes active during the structural dynamics,
greatly reducing the dimensionality of the problem herein.

% Diarylethene
In Ch.~\ref{ch: UED-DAE}, UED was performed on single crystals of a diarylethene derivative
to follow the structural dynamics of the photocyclization reaction
and the formation of the closed-ring isomer.
%
The exact moment of the bond-forming event was observed with femtosecond and sub-angstrom resolution.
Despite the many available degrees of freedom in the system,
the results showed that the ring-closing process also involved only a small number
of atomic motions, suggesting a similar reduction in dimensionality as in (EDO-TTF)$_2$X.

% SCO
% $\mathrm{Fe^{II}(PM-AzA)_2 (NCS)_2]}$, and .
Later, the photoinduced spin crossover process in two transition metal complexes
--- $\mathrm{[Fe^{II}(bpy)_3](PF_6)_2}$ (BPY) and $\mathrm{[Fe^{II}(PM-AzA)_2 (NCS)_2]}$ (AZA)
--- was studied with femtosecond transient absorption spectroscopy and UED.
These results were described and discussed in Ch.~\ref{ch: SCO},
and Secs.~\ref{sec: freq-anal-BPY} and~\ref{sec: UED-BPY}.
%
From the BPY~transient absorption data, the ultrafast electronic dynamics of spin crossover in the solid state
was shown to be comparable with that in the solution phase, as measured here and
reported elsewhere.
%
On this basis, the first direct characterization of the structural dynamics associated with
photoinduced spin crossover was made by UED in single crystals of AZA
using near-uniform low-fluence sample excitation. From this data, it was found that
all the resolvable atomic motions could be robustly fitted by a single monoexponential function
with time constant ca.~$2.3$~ps and a 3D~structural model involving
Fe--N bond elongation, ligand motion, and unit~cell expansion.
%
Preliminary UED results in single-crystal~BPY provided further support to
this ligand-inclusive interpretation.

Ch.~\ref{ch: future} describes in-progress works that I have carried out.

% Finally
In summary, the experimental and analytical results that were presented
in the pages of this thesis constitute my unique and original contribution to the body of work
that is at the intersection of chemical physics and ultrafast science.
%
Although many more questions remain to be answered, at least one ---
making molecular movies of non-trivial systems and dimensionality reduction ---
has been dealt with and some of the the rules that govern the `dancing of atoms' are now
much clearer.


% Epilogue
\newpage
\renewcommand{\epigraphflush}{center}
\renewcommand{\sourceflush}{flushright}
% \setlength\epigraphrule{0pt}
\setlength\epigraphwidth{0.6\textwidth}
\vspace*{\fill}
\epigraph{
I could tell innumerable other stories \\
and they would all be true: \\
all literally true, \\
in the nature of the transitions, \\
in their order and data. \\
The number of atoms is so great \\
that one could always be found \\
whose story coincides \\
with any capriciously invented story.}%
{\vspace{0.25\baselineskip}
\textsc{Primo Levi} \\
``Carbon,'' \textit{The Periodic Table}, p.~232~\cite{LeviBook}}


%
